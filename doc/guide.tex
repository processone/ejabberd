\documentclass[12pt]{article}

%\usepackage{graphics}
\usepackage{hevea}
\usepackage{verbatim}


\newcommand{\ejabberd}{\texttt{ejabberd}}
\newcommand{\Jabber}{Jabber}

\newcommand{\modregister}{\texttt{mod\_register}}
\newcommand{\modroster}{\texttt{mod\_roster}}
\newcommand{\modconfigure}{\texttt{mod\_configure}}
\newcommand{\moddisco}{\texttt{mod\_disco}}
\newcommand{\modstats}{\texttt{mod\_stats}}
\newcommand{\modvcard}{\texttt{mod\_vcard}}
\newcommand{\modoffline}{\texttt{mod\_offline}}
\newcommand{\modecho}{\texttt{mod\_echo}}
\newcommand{\modprivate}{\texttt{mod\_private}}
\newcommand{\modtime}{\texttt{mod\_time}}
\newcommand{\modversion}{\texttt{mod\_version}}



\title{Ejabberd Installation and Operation Guide}
\author{Alexey Shchepin \\
  \ahrefurl{mailto:alexey@sevcom.net} \\
  \ahrefurl{xmpp:aleksey@jabber.ru}}
\date{January 23, 2003}

\begin{document}
\begin{titlepage}
  \maketitle{}
  


  %\includegraphics{logo.png}
  {\centering
    \imgsrc{logo.png}{}
  }
\end{titlepage}
%\newpage
\tableofcontents{}

\newpage
\section{Introduction}
\label{sec:intro}

\ejabberd{} is a Free and Open Source distributed fault-tolerant \Jabber{}
server.  It writen mostly in Erlang.

TBD



\section{Installation}
\label{sec:installation}


\subsection{Installation Requirements}
\label{sec:installreq}

To compile \ejabberd{}, you need following packages:
\begin{itemize}
\item GNU Make;
\item GCC;
\item libexpat 1.95 or later;
\item Erlang/OTP R8B or later.
\end{itemize}

\subsection{Obtaining}
\label{sec:obtaining}

Currently no stable version released.

Latest alpha version can be retrieved via CVS.  Do following steps:
\begin{itemize}
\item \texttt{export CVSROOT=:pserver:cvs@www.jabber.ru:/var/spool/cvs}
\item \texttt{cvs login}
\item Enter empty password
\item \texttt{cvs -z3 co ejabberd}
\end{itemize}






\subsection{Compilation}
\label{sec:compilation}




%\subsection{Initial Configuration}
%\label{sec:initconfig}


\section{Configuration}
\label{sec:configuration}

\subsection{Initial Configuration}
\label{sec:initconfig}

%\verbatiminput{../src/ejabberd.cfg}

Configuration file is loaded after first start of \ejabberd{}.  It consists of
sequence of Erlang terms.  Parts of lines after \texttt{`\%'} sign are ignored.
Each term is tuple, where first element is name of option, and other are option
values.


\subsubsection{Host Name}
\label{sec:confighostname}

Option \texttt{hostname} defines name of \Jabber{} domain that \ejabberd{}
serves.  E.\,g. to use \texttt{jabber.org} domain add following line in config:
\begin{verbatim}
{host, "jabber.org"}.
\end{verbatim}

This option is mandatory.



\subsubsection{Listened Sockets}
\label{sec:configlistened}

Option \texttt{listen} defines list of listened sockets and what services
runned on them.  Each element of list is a tuple with following elements:
\begin{itemize}
\item Port number;
\item Module that serves this port;
\item Function in this module that starts connection (likely will be removed);
\item Options to this module.
\end{itemize}

Currently three modules implemented:
\begin{itemize}
\item \texttt{ejabberd\_c2s}: serves C2S connections;
\item \texttt{ejabberd\_s2s\_in}: serves incoming S2S connections;
\item \texttt{ejabberd\_service}: serves connections to \Jabber{} services (i.e.
  that used \texttt{jabber:component:accept} namespace).
\end{itemize}

For example, following configuration defines that C2S connections listened on
port 5222, S2S on port 5269 and that service \texttt{conference.jabber.org}
must be connected to port 8888 with password ``\texttt{secret}''.

\begin{verbatim}
{listen, [{5222, ejabberd_c2s,     start, []},
          {5269, ejabberd_s2s_in,  start, []},
          {8888, ejabberd_service, start, ["conference.jabber.org", "secret"]}
         ]}.
\end{verbatim}


\subsubsection{Access Rules}
\label{sec:configaccess}

TBD


\subsubsection{Modules}
\label{sec:configmodules}

Option \texttt{modules} defines list of modules that will be loaded after
\ejabberd{} startup.  Each list element is a tuple where first element is a
name of module and second is list of options to this module.  Refer to
section~\ref{sec:modules} for detailed information on each module.

Example:
\begin{verbatim}
{modules, [
           {mod_register,  [one_queue]},
           {mod_roster,    [one_queue]},
           {mod_configure, [one_queue]},
           {mod_disco,     [one_queue]},
           {mod_stats,     [one_queue]},
           {mod_vcard,     [one_queue]},
           {mod_offline,   []},
           {mod_echo,      []},
           {mod_private,   [one_queue]},
           {mod_time,      [one_queue]},
           {mod_version,   [one_queue]}
          ]}.
\end{verbatim}


\subsection{Online Configuration}
\label{sec:onlineconfig}

To use facility of online reconfiguration of \ejabberd{} needed to have
\modconfigure{} loaded (see section~\ref{sec:modconfigure}).  Also highly
recommended to load \moddisco{} (see section~\ref{sec:moddisco}), because
\modconfigure{} highly integrates with it.  Also recommended to use
disco-capable client.

TBD

\section{Distribution}
\label{sec:distribution}





\section{Built-in Modules}
\label{sec:modules}




\subsection{\modregister{}}
\label{sec:modregister}



\subsection{\modroster{}}
\label{sec:modroster}



\subsection{\modconfigure{}}
\label{sec:modconfigure}



\subsection{\moddisco{}}
\label{sec:moddisco}



\subsection{\modstats{}}
\label{sec:modstats}



\subsection{\modvcard{}}
\label{sec:modvcard}



\subsection{\modoffline{}}
\label{sec:modoffline}



\subsection{\modecho{}}
\label{sec:modecho}



\subsection{\modprivate{}}
\label{sec:modprivate}



\subsection{\modtime{}}
\label{sec:modtime}



\subsection{\modversion{}}
\label{sec:modversion}






\end{document}
